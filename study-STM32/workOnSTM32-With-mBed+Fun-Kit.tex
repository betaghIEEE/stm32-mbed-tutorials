% !TEX TS-program = xelatex
%
% Created by Daniel Beatty on 2020-09-25.
% Copyright (c) 2020 All Night Star Party.
\documentclass{article}

\usepackage{polyglossia}
\usepackage{hyperref}
\usepackage{listings}
\usepackage{algorithm,algorithmic}

\hypersetup
{
  pdftitle   = {Working Electrical Engineering Projects with STM32 + Fun Kit},
  pdfsubject = {Electrical Engineering, Computer Science, Embedded Computing},
  pdfauthor  = {Daniel Beatty}
}

\title{Working Electrical Engineering Projects with STM32 + Fun Kit}
\author{Daniel Beatty}

\begin{document}

\maketitle

\begin{abstract}
    Abstract
\end{abstract}

\section{Introduction}

The work developed in this document focuses on the needs of one of my clients.  He needed a micro-controller to manage a device to drive large power loads.  This document therefore focuses on one of the most primal features of the internet of things.  This work falls in a series that I call `Back to the Basics.'

%Why am I doing this?  I have some examples provided by a client of mine.  These might have worked in some STM32 development environments.  However, software evolves to meet hardware and security needs.  I suspect that I am not alone.  
So, what do I want out of a back to the basics book?  
I would prefer to have a cross platform means to develop software. I would prefer the ability to analyze the stacks deployed on these devices from the kernel up.  I would like to ensure that I can deliver a good product, and apply micro and macro architectures (aka design patterns) as well good algorithms.  

To do this, this work focuses on the mBed system of development.  This document shows this project utilizing the STM32 product line.  In some cases, I also utilize the Arduino products just to demonstrate a cross manufacture approach to testing designs.  Lastly, this document shows how to do this work on the MacOS, Windows, and Linux (ARM) platform.  
%So, I want to know how this mBed system work.  I will start with the STM32 product line to develop this micro-architecture approach.  


\newpage
\section{The Base Line Integrated Development Environment Part 1} % (fold)
\label{sec:the_base_line_integrated_development_environment_part_1}

The Eclipse Foundation and Advanced RISC Machines(ARM) consortium work together to produce an Integrated Development Environment(IDE)\footnote{https://os.mbed.com} to fulfill needs of micro-controller developers.  The development community needs build services and IDE systems to handle the complex builds. %that include the real-time operating system (RTOS) itself and supplimental main purpose software.  %This capability follows in the footsteps of many micro-controllers that came before, and enables a new generation of makers.

The Advanced RISC Machines (ARM) Limited developed the mBedOS to exhibit standard features amongst various manufactures builds.  Thus, I can develop a simple program and compile it for many different boards, and it still works. 

Mbed Studios (a modified Eclipse Theia) released in June 2019.  It is derived from the Eclipse `Theia' line shown in a YouTube video\footnote{https://www.youtube.com/watch?v=HsTtzqL-GP8}.

ARM made the mBed Suite \footnote{https://os.mbed.com/studio/} based on Eclipse Theia  \footnote{https://youtu.be/NLkQzx6rrnU \\https://www.youtube.com/channel/UCXu7pV552EkR99mCzjwGvTQ/featured}.  This IDE supports inclusion of the mBed OS itself in a build and any additional libraries required to construct a software product for the ARM based boards.  These libraries include:
\begin{itemize}
	\item `Tiny interactions'
	\item Communication mechanisms
	\item Hardware Abstraction Layer(s)
\end{itemize}



ARM supplies a tutorial on installing the mBed Theia environment on MS Windows 10\footnote{https://os.mbed.com/teams/ST-Americas-mbed-Team/wiki/Getting-Started-with-mbed-and-the-STM32F}.  We will examine Mac and Windows on Intel processor machines.   For example, I develop embedded software on a MacBook Pro and also demonstrate with a Mac Min running MS Windows with bootcamp.  This work also seeks to produce Raspberry Pi and iPad tools to augment the STM Micro-controllers.  %\textbf{I} will also attempt to set up the mBed Studio on a Raspberry Pi.   What would be useful is to establish a Jenkins + Version control + Testing + Distribution suite for these applications.  


\subsection{ST Microelectronics Libraries} % (fold)
\label{sub:st_microelectronics_libraries}

ST Microelectronics started as a French company.  It operates out of many nations.  In addition to producing the micro-controller boards themselves, they also produce support libaries for their boards\footnote{http://www.emcu.it/STM32F4xx/STM32F4-Library/STM32F4-Library.html}.  For example, ST Microelectronics is not the first manufacturer to produce ARM based digital signal processor (DSP)\footnote{https://www.st.com/content/st\_com/en/products/embedded-software/mcu-mpu-embedded-software/stm32-embedded-software/stm32-standard-peripheral-libraries/stsw-stm32065.html}.  DSP engines provide a similar acceleration to vector / array math operations that a graphics processor units (GPU) provides to two and three dimensional arrays.   In some cases, GPU systems provide the DSP and general purpose accelerated capabilities all in one facility. 






% subsection st_microelectronics_libraries (end)

\subsection{Hardware Abstraction Layer(s)} % (fold)
\label{sub:hardware_abstraction_layer_s}

ST Microelectroincs and mbedOS supplies Hardware Abstraction Layer(s) to access key MCU capabilites.  The mbedOS is not the only real time operating system (RTOS) for micro-controllers.  ARM's mbedOS does support many implementations of its chip set \footnote{https://os.mbed.com/questions/53876/CMSIS-vs-STM32CUBEHAL-vs-MBED/} and builds into a tidy size for each set it supports.




The mBed website carries white paper references for each of the boards it supports.  For example, this piece focuses on the Nucleo-F413ZH\footnote{https://os.mbed.com/platforms/ST-Nucleo-F413ZH/}.  This platform and others can be found at \footnote{https://os.mbed.com/platforms/}.
Alternative cards include:
\begin{itemize}
	\item Discovery F413H \footnote{https://os.mbed.com/platforms/ST-Discovery-F413H/}
	\item Nucleo L4R5ZI-P \footnote{https://os.mbed.com/platforms/ST-Nucleo-L4R5ZI-P/}
	\item Nucleo WB55RG \footnote{https://os.mbed.com/platforms/ST-Nucleo-WB55RG/}
\end{itemize}

Some of the main library sources from STM are provided on the STM32f4 Discovery web site\footnote{http://stm32f4-discovery.net/2014/05/all-stm32f429-libraries-at-one-place/} forum.  

%The Nucleo-F413ZH does not have a direct ethernet connection, but resources like this GitHub repository \footnote{https://github.com/khoih-prog/EthernetWebServer\_STM32/tree/master/src/detail} suggest that there are ways to make it work.

% subsection hardware_abstraction_layer_s (end)

\subsection{It is about the Libraries, Shhsss.} % (fold)
\label{sub:it_is_about_the_libraries_shhsss}

For any task associated with a micro-controller, ARM based micro-controllers require the libraries.  ARM micro-controllers require the frameworks of the RTOS itself, in this case mBedOS.  

Developers refactor their code as they get into development.  These libraries serve as building blocks.  This product explores basic library creation to benefit fellow developers and makers of products.

As we create libraries and programs, it behoves us to consider license programs and how to market our productions.  We want the fruits of our labor to bring us profit.  It might if we can attract people to use it and honor our contributions.  It certainly won't if we hoard it or never make it.  

So we explore creating programs, importing existing libraries, and creating new libraries.  Documentation of such products contributes to fellow makers utilizing our work to make things and solving problems.  We make money from establishing those connections and making the software that provide things their connection to real life.

\subsubsection{Create a Repository} % (fold)
\label{ssub:create_a_repository}

We present software creation from a independent producers point of view.  Many corporations and government entities insist on their own repository tools.  For these entities, we can comprehend the need of such restrictions.  They have assests to protect and the finances to provide on property protection.  

% subsubsection create_a_repository (end)

\subsubsection{Creating A Program} % (fold)
\label{ssub:creating_a_program}
The mBed Suite documentation provides an excellent how-to guide on creating a program \footnote{https://os.mbed.com/docs/mbed-studio/current/create-import/index.html}.   This process works out in a very straight forward way.  



% subsubsection creating_a_program (end)

\subsubsection{Importing A Program} % (fold)
\label{ssub:importing_a_program}

% subsubsection importing_a_program (end)

\subsubsection{Importing Libraries} % (fold)
\label{ssub:importing_libraries}


\footnote{https://os.mbed.com/docs/mbed-studio/current/manage-libraries/index.html#importing-a-library}

% subsubsection importing_libraries (end)

\subsubsection{Creating Libraries} % (fold)
\label{ssub:creating_libraries}

% subsubsection creating_libraries (end)


% subsection it_is_about_the_libraries_shhsss (end)


% section the_base_line_integrated_development_environment_part_1 (end)



\section{Basic Hard Tools} % (fold)
\label{sec:basic_hard_tools}

\subsection{Work on Timer} % (fold)
\label{sub:work_on_timer}

In cases where we ask the process to wait, we should ensure that the I/O pipe empties.  Otherwise, we have no idea when this process takes place.  It can occur in the wrong times.

There are third party USB drivers such as \footnote{https://os.mbed.com/users/gte1/code/STM32\_USBDevice/}.

Also, the timer's have issue with floating point numbers.

Lastly, the STDIO has trouble with printing floats.   

% subsection work_on_timer (end)

\subsection{Flushing The Standard Input Output} % (fold)
\label{sub:flushing_the_standard_input_output}
Prior to mBedOS 5, the use of the C Standard Input Output library meant opening a USB serial port.  Most of the C Standard Input Output library functions were methods of these singleton classes.  

With mBedOS 6, the C Standard Input Output library uses the main USB serial port by default.  Therefore, serial communication works the way a typical desktop/laptop environment does, with one exception.  Threads and Interrupt Service Routines (especially timers) don't play nice with the buffer.  The process of printing or scanning characters does not have the real time behavior when the either real time events occur.  Thus, the developer does have to keep track to clear the buffer.

% subsection flushing_the_standard_input_output (end)


\subsection{Blinky} % (fold)
\label{sub:blinky}

For many in computer programming industries, the first program a developer writes is `hello world.'  The goal of such a program is to demonstrate that when developer runs the program that it does something the user can observe.  

In this example, we develop a program called blinky.  This programs main objective seeks to show some observable work.  In this case, the developer can put together a external circuit or use on board lights of the Nucleo to show the program doing work.  

\begin{algorithmic}
	\STATE Setup DigtialOut Class for pin $PC\_1$
	\STATE Setup DigtialOut Class for pin $PC\_0$ 
	\WHILE {Always \TRUE}
		\STATE Toggle $PC\_1$
		\STATE Wait 
		\STATE Toggle $PC\_0$
		\STATE Wait
	\ENDWHILE
\end{algorithmic}

We can also do basic threads to allow the OS to handle wake up times.  The basic algorithm for the thread control function is similar.  The pin object needs to be relatively global to the name space.  A good exercise is to build a class for blinky with pin and timer value for members.  The actions of this class need to include its initiation with pin number id and default timer values.  More actions entail starting and stopping the thread.  

The ARM team changed the thread library to use the standard chrono space for its time measurements.  For those of us who trained first on C/C++ 99 and earlier, use of this library seems foriegn.  As shown in the listing, an object can be created from this Standard Chrono namespace.  mBed Studio also allows compound interpretation of number and unit as the same namespace object as shown in the main body.

\begin{lstlisting}
	#include "PinNames.h"
	#include "mbed.h"
	#include "rtos.h"
	#include <chrono>
	#include <ratio>

	DigitalOut LEDA (PC_1);
	DigitalOut LEDB(PC_0);
	Thread thread1, thread2;

	void LEDAControl()
	{
	    while(true){
	        LEDA = !LEDA;
	        std::chrono::seconds d(1);
	        ThisThread::sleep_for(d);
	    }
	}



	void LEDBControl()
	{
	    while(true){
	        LEDB = !LEDB;
	        std::chrono::milliseconds d(500);
	        ThisThread::sleep_for(d);
	    }
	}

	// main() runs in its own thread in the OS
	int main()
	{
	    thread1.start(LEDAControl);
	    thread2.start(LEDBControl);
	    while (true) {
	        ThisThread::sleep_for(10s);
	    }
	}
\end{lstlisting}

% subsection blinky (end)

\subsection{Seven Segment LED} % (fold)
\label{sub:seven_segment_led}

% Reference ARM Book Chapter 7  section 16
The seven segment light emitting diode circuit example yields some interesting issues for a microcontroller.  One, the basic digit requires one digital out pin per segment (in other words seven).   Two, the circuit also requires a common ground.  Three, if the circuit multiple digits involves multiple digits, these devices store logic levels.  This allows another digital pin out set to determine which digit to change.  

This issue is a classic issue dating back to early micro-controllers.  This issue allows most instructors of micro-controller theory to segway a lesson from basic digital design into the curriculum.  Therefore, we see a simple software definition of a seven segment LED structure to determine the digital output pins.

The mBedOS boards allow software to address whole port families.  We can address ports A, B, C, etc. as a whole register.  This makes the development of libraries and classes a much simpler exercise.  

One word of advice, pick your use of ports carefully.  If the controller needs to read in analog to digital, produce pulse width modulation, or digital to analog out then these operations can't share the port devoted to managing the seven segment LED.

The basic digital design portion of this exercise requires one to map the each digit to what pins must be turned on.  This map must then be applied to an 8-bit word that controls the pins.  Note, the designer can do this with individual digital out operations.  It simply requires on the order of 8 sets operations more to do so by individual digital out instructions.  Where as, assigning a single 8-bit word requires fewer instructions.

In this example, we use 5161AS seven segment LED provided in the Rex Qualis Uno R3 Project Starter Kit.  There are other seven segment LED models made, and one must identify the segments to build the circuit properly. 

The pins for the Nucleo-F413ZH can be found online\footnote{https://os.mbed.com/platforms/ST-Nucleo-F413ZH/}.  The pins for a basic bread board with the 5161AS seven segment LED can be found in figure \ref{}.  The port settings to equal the number to be displayed are found in table \ref{}.

\begin{table}[htb!]
	\caption{Seven Segment Display Port Values}
	\label{table:label}
	\centering
	\begin{tabular}{llllllllll}
		\toprule
		\textbf{Number} & \textbf{PC 7} & \textbf{PC 6} & \textbf{PC 5} & \textbf{PC 4} & \textbf{PC 3} & \textbf{PC 2} & \textbf{PC 1} & \textbf{PC 0} & \textbf{Hex Value}\\
		\midrule
		             0 &              0 &              0 &              1 &              1 &              1 &              1 &              1 &              1 &              0x3F\\
		             1 &              0 &              0 &              0 &              0 &              0 &              1 &              1 &              0 &              0x06\\
		             2 &              0 &              1 &              0 &              1 &              1 &              0 &              1 &              0 &              0x5b\\
		             3 &              0 &              1 &              0 &              0 &              1 &              1 &              1 &              1 &              0x4F\\
		             4 &              0 &              1 &              1 &              0 &              1 &              1 &              1 &              0 &              0x66\\
		             5 &              0 &              1 &              1 &              0 &              1 &              1 &              0 &              1 &              0x6D\\
		             6 &              0 &              1 &              1 &              1 &              r5c6 &              1 &              0 &              1 &              0x7D\\
		             7 &              0 &              0 &              0 &              0 &              0 &              1 &              1 &              1 &              0x07\\
		             8 &              0 &              1 &              1 &              1 &              1 &              1 &              1 &              1 &              0x7f\\
		             9 &              0 &              1 &              1 &              0 &              1 &              1 &              1 &              1 &              0x6f\\
		\bottomrule
	\end{tabular}
\end{table}

\begin{algorithmic}
	\STATE Configure Port C Lower Byte as a group
	\STATE Define segment patterns
	\STATE Set Count to 0  
	\WHILE {Always \TRUE}
		\STATE Send count to display
		\STATE Wait 1 second
		\IF {Count is greater or equal to 10} 
			\STATE set count to 0
		\ENDIF
	\ENDWHILE
\end{algorithmic}

\begin{lstlisting}
/*
	This 7-Segment Counter is based on a textbook 
example from Dogan Ibrahim's ARM-based Microcontroller Projects Using mbed.

	Modifications were made by Dan Beatty to adjust to mBed Studio and mBedOS 6.x.
*/


#include "mbed.h"
#include "rtos.h"

PortOut Segments(PortC, 0xFF);
int LEDS[] = {0x3f, 0x06, 0x5b, 0x4f, 0x66, 0x6d, 0x7d, 0x07, 0x7f, 0x6f};

// main() runs in its own thread in the OS
int main()
{
	int count = 0;
	
    while (true) {
		Segments = LEDS[count];
		wait_us(1000000);
		count++;
		if (count >= 10) count = 0;
        printf("%d \n", count);

    }
}
\end{lstlisting}




% subsection seven_segment_led (end)


\subsection{Basic Digital Input} % (fold)
\label{sub:basic_digital_input}

%  Reference ARM Book Chapter 7 section 13 - 14
Basic digital input can be demonstrated by an 3 sets of switch input.  The input goes low when the switch is closed and high when the switch is open.  The switches are simple buttons provided in the Rex Qualis Electronic Component Fun Kit.  This kit also has an RGB LED.  The top of the LED is transparent and one can see the etching.  The fat etching is the common ground.  We connect the other three to red, blue, and green control pins.  These pins need a 330/390 ohm resistor to establish sufficient current limits for the Nucleo Board.

\begin{algorithmic}
	\STATE Configure the RGB output
	\STATE Configure the R, G, B buttons as inputs and enable them as Pull-ups
	\STATE Clear all LEDs
	\WHILE{Forever TRUE}
		\IF {R is pressed}
		\STATE Toggle Red color for display
		\ENDIF
		\IF {G is pressed}
		\STATE Toggle Green color for display
		\ENDIF
		\IF {B is pressed}
		\STATE Toggle Blue color for display
		\ENDIF
	\ENDWHILE
\end{algorithmic}

\begin{lstlisting}
	/*********************************************
	In this project, an RGB LED is connect. So are 3 buttons. 
	Pressing a button toggles the corresponding color.

	R button = PC_0
	G button = PC_1
	B button = PC_2

	The RGB LED pin are connected as follows:
	Red = PC_3
	Green = PC_4
	Blue = PC_5

	Original Supplied by Dogan Ibrahim, August 2018 in 
	"ARM-based Microcontroller Projects Using mBed"
	Modified by Dan Beatty for use in mBed Studio 
	mBed OS 6.x
	
	*********************************************/


	#include "mbed.h"

	BusOut RGB(PC_3, PC_4, PC_5);
	DigitalIn R(PC_0, PullUp);
	DigitalIn G(PC_1, PullUp);
	DigitalIn B(PC_2, PullUp);

	// main() runs in its own thread in the OS
	int main()
	{
		RGB = 0;
	    while (true) {
			if (R == 0) {
				if (RGB & 1 == 1)
					RGB = RGB & 6;
				else
					RGB = RGB | 1;
			}
			if (G == 0) {
				if (RGB & 2 == 2)
					RGB = RGB & 5;
				else
					RGB = RGB | 2;
			}
			if (B == 0) {
				if (RGB & 4 == 4)
					RGB = RGB & 3;
				else
					RGB = RGB | 4;
			}
	    }
	}
\end{lstlisting}

% subsection basic_digital_input (end)



\subsection{Relays} % (fold)
\label{sub:relays}

%  Reference ARM Book Chapter 7-17

% subsection relays (end)


\subsection{Optical Isolators} % (fold)
\label{sub:optical_isolators}

%

% subsection optical_isolators (end)

% section basic_hard_tools (end)


\section{Analog to Digital Conversion} % (fold)
\label{sec:analog_to_digital_conversion}

To test the basic idea of a Analog to Digital Conversion (ADC), we can use the concept of a voltmeter.  Another way is with a thermistor circuit.  Some temperature sensitive devices have a more calibrated response and we have formulae that can translate the voltage to known temperature scales easily.   

\subsection{Simple Analog In - Voltmeter} % (fold)
\label{sub:simple_analog_in_voltmeter}

This example sets up two versions of the same general circuit.  The first version is the simplest one can make.  Often, this version does not scale to real world problems.  Real world problems can include large power issues that would damage the microcontroller in the process.  The second version compensates for circuits with larger currents by placing a optical isolator in between the test circuit and the microcontroller.  


%We can make a simple analog example using the parts in the Fun Kit.   We design the circuit to have very little current.  

I learned to look up the pin name and its location on the board.   There pin names that don't necessarily correspond to the label next pin itself.  The A0 pin on the CN9 pin set is actually PA\_3.  It is easy to make the mistake in calling this pin PA\_0.   This distinction can lead to massive mistakes and cause to operate the board in a wrong manner.  

To demonstrate simple ADC in, I use a simple voltage divider circuit with three resistors.  Voltage is proportial to resistance.  Therefore, we can measure voltage across a resistor and expect its value.  That proportion should change if we can the resistance of one of the other resistors.  

Therefore, I establish this measurement across a constant 330 $\omega$ resistor.  Then, I do similar across a 330 $\omega$ potentiometer.  We see the voltage go from $V_{in}$.  This helps establish a baseline for comparing the ADC readings.   

I built the circuit with parts acquired from \footnote{http://www.rexqualis.com/products/}.  The fun kit \footnote{http://www.rexqualis.com/product/electronics-component-fun-kit-w-power-supply-module-male-to-female-jumper-wire-830-tie-points-breadboard-precision-potentiometer-resistor-for-arduino-raspberry-pi-stm32/} 
provides many of the resistors I need.   The UNO Project starter \footnote{http://www.rexqualis.com/product/uno-project-super-starter-kit-for-arduino-w-uno-r3-development-board-detailed-tutorial/} provides some addition parts that I can use later.  Also the UNO Project starter provides a Arduino UNO that I can use for comparison.  Lastly, I used some competitor products to provide the potentiometers and inductors that I need for future examples.  


Also, it is important to note a fix on stdio.  Mbed.os by default turns off floating point parameters in stdio/UART\footnote{https://github.com/ARMmbed/mbed-os/blob/master/platform/source/minimal-printf/README.md}.  
There is a parameter in platform/mbed\_app.json.  By default, a new mbed project has the value for `minimum-printf-enable-floating-point' set to false.  In order to print out floating point, this value needs to be set to true.  

With this, we can see in the example that the voltage read from the potenimeter is pretty close to what a common multimeter reads.  There are adjustment factors that we apply to give a rough calibration.  For production, it would help to obtain a more calibrated set of values.  

We see a similar case with thermister based thermometers.  I set up a simple $330\omega$ resistor in series with a thermister out of the fun kit.  Here we can measure the voltage across the thermister.  We then can measure ratios and offsets for temperature.  
\begin{equation}
	T = \frac{(mV - 225.0)}{10.0}
\end{equation}

% subsection simple_analog_in_voltmeter (end)


\subsection{Change Flashing Rate with Potentiometer} % (fold)
\label{sub:change_flashing_rate_with_potentiometer}



% subsection change_flashing_rate_with_potentiometer (end)

\subsection{Thermostat} % (fold)
\label{sub:thermostat}

% subsection thermostat (end)

\subsection{Light Meter} % (fold)
\label{sub:light_meter}

% subsection light_meter (end)

\subsection{Thermo-resistor -Taking the Temperature} % (fold)
\label{sub:thermo_resistor_taking_the_temperature}

This is an example of an Analog in problem


% subsection thermo_resistor_taking_the_temperature (end)

\subsection{Sound Level Meter} % (fold)
\label{sub:sound_level_meter}

% subsection sound_level_meter (end)


% section analog_to_digital_conversion (end)


\section{Analog Out} % (fold)
\label{sec:analog_out}

\subsection{Fixed Voltage} % (fold)
\label{sub:fixed_voltage}

% subsection fixed_voltage (end)

\subsection{Sawtooth} % (fold)
\label{sub:sawtooth}

% subsection sawtooth (end)

\subsection{Sine Wave Generator} % (fold)
\label{sub:sine_wave_generator}

% subsection sine_wave_generator (end)


\subsection{Combining the Capabilites of Each Wave Generator Type} % (fold)
\label{sub:combining_the_capabilites_of_each_wave_generator_type}

% subsection combining_the_capabilites_of_each_wave_generator_type (end)

\subsection{Buzzer Fun} % (fold)
\label{sub:buzzer_fun}

% subsection buzzer_fun (end)

\subsection{Pulse Width Modulated Buzzers} % (fold)
\label{sub:pulse_width_modulated_buzzers}

% subsection pulse_width_modulated_buzzers (end)

% section analog_out (end)


\section{Optical Isolation} % (fold)
\label{sec:optical_isolation}

I remember a lesson from my embedded computing class back in the Spring of 1998.  Professors Darrel Vines, Michael Parton, and Mike Giesselman told us over and over about the importance of optical isolation.  These professors worked in pulsed power, and I wish I had the wisdom back then to realize just how important this subject is.  % and literally how much power goes into such devices.  

Pulsed Power delivers a large amount of current in a very short space and time.  The amount of current involved can easily melt a microcontroller and destroy it.  Yet, these controllers are essential in regulating such power elements to fulfill their purpose.  

The fun kit comes with a 4N35 white paper can be found at\footnote{https://www.digchip.com/datasheets/parts/datasheet/161/4N35-pdf.php}. The inputs on pins 1 and 2 can be thought of as in and out for a light emitting diode (LED).  The light from the LED feeds a photo-transistor that can handle much higher currents. 

% section optical_isolation (end)

\section{Transistor Circuits} % (fold)
\label{sec:transistor_circuits}

PN2222

Transistor


% section transistor_circuits (end)


\section{Ethernet Web Server Libraries} % (fold)
\label{sec:ethernet_web_server_libraries}

\footnote{https://github.com/khoih-prog/EthernetWebServer\_STM32}



% section ethernet_web_server_libraries (end)

\section{Digital Signal Processing} % (fold)
\label{sec:digital_signal_processing}

\footnote{http://www.emcu.it/STM32F4xx/STM32F4-Library/STM32F4-Library.html}




% section digital_signal_processing (end)


\section{Secure Digital Block Devices} % (fold)
\label{sec:secure_digital_block_devices}

%\footnote{https://github.com/Ralim/stm32-libraries}

% section secure_digital_block_devices (end)


\section{Home Made RADAR} % (fold)
\label{sec:home_made_radar}


%\footnote{https://os.mbed.com/users/vhx/code/Nucleo\_radar/}

% section home_made_radar (end)

\section{Registering the Serial Number} % (fold)
%\label{sec:registering_the_serial_number}

%\footnote{https://forums.mbed.com/t/accessing-device-serial-number-stm32f4xx-chips-for-example/8120}

% section registering_the_serial_number (end)


\section{Multiplexed LED Multi-Thread} % (fold)
\label{sec:multiplexed_led_multi_thread}

The original work for this example comes from \cite{Ibrahim-ARM-based-Microcontrollers}.  The value in this exercise exists in the multi-thread operations.   We use this lesson to construct classes that inherently use threads to gather information.

% section multiplexed_led_multi_thread (end)

\end{document}
